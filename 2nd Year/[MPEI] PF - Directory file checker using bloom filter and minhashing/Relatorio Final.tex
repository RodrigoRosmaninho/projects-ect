\documentclass{report}
\usepackage[T1]{fontenc} % Fontes T1
\usepackage[utf8]{inputenc} % Input UTF8
\usepackage{csquotes}
\usepackage[portuguese]{babel} %Usar língua portuguesa
\usepackage{blindtext} % Gerar texto automaticamente
\usepackage[printonlyused]{acronym}
\usepackage{hyperref} % para autoref
\usepackage{graphicx}

\begin{document}
%%
% Definições
%
\def\titulo{Trabalho Prático}
\def\data{Dezembro de 2018}
\def\autores{Rodrigo Rosmaninho, André Alves - Turma P5}
\def\autorescontactos{(88802) r.rosmaninho@ua.pt, (88811) andr.alves@ua.pt}
\def\versao{Relatório Final}
\def\departamento{DETI - Universidade de Aveiro}
\def\empresa{Métodos Probabilísticos para Engenharia Informática}
\def\logotipo{ua.pdf}
%
%%%%%% CAPA %%%%%%
%
\begin{titlepage}
\begin{center}
%
\vspace*{40mm}
%
{\Huge \titulo}\\ 
%
\vspace{10mm}
%
{\Large \empresa\\\departamento}
%
\vspace{8mm}
%

{\Large \autores \\
    \autorescontactos \\
}
%
\vspace{8mm}
%
{\Large \data}\\
%
\vspace{24mm}
%
\begin{figure}[h]
\center
\includegraphics{\logotipo}
\end{figure}
%
\vspace{30mm}
\end{center}
%
\begin{flushright}
\versao
\end{flushright}
\end{titlepage}

\clearpage
\pagenumbering{arabic}


%%%%%%%%%%%%%%%%%%%%%%%%%%%%%%%%
\chapter{Módulos e Aplicação de Uso Conjunto}
\label{chap.Modulos}
\section{Aplicação 'Directory Tools'}
Para demonstrar os módulos a trabalharem em conjunto, foi desenvolvida uma aplicação com interface visual (Java Swing) e que permite:
\begin{itemize}
    \item Escolher um diretório no sistema cujos ficheiros vão ser lidos e adicionados ao Counting Bloom Filter após terem sido geradas as assinaturas de cada um (através do algoritmo de MinHash).
    \item Obter a soma de todos os bytes lidos (ou seja, o tamanho do diretório lido) através de contagem estocástica.
    \item Escolher um ficheiro e verificar se existe uma cópia deste no diretório que foi lido, mesmo que os nomes sejam diferentes (verificar presença no bloom filter). Mesmo que não exista é indicado, quando possível, qual o ficheiro do diretório lido é mais semelhante ao ficheiro escolhido (Distância de Jaccard com Locality Sensitive Hashing).
    \item Detetar e apresentar eventuais ficheiros duplicados no diretório lido, mesmo que tenham nomes diferentes (Distância de Jaccard com Locality Sensitive Hashing).
    \item Visualizar todos os ficheiros lidos, em forma de lista.
    \item Eliminar ficheiros do Counting Bloom Filter, para que estes deixem de ser contemplados em operações subsequentes ou exibidos na lista de ficheiros. (função remove())
    \item Contar o número de vezes que um determinado ficheiro foi adicionado ao Counting Bloom Filter. (função count())
\end{itemize}

\begin{center}
\includegraphics[scale=0.43]{dirTools.png}
\end{center}

Recomedam-se diretórios com tamanho não superior a 20MB de forma a minimizar o tempo de espera e a utilização de memória ao ser efetuada a leitura, adição ao Bloom Filter, e cálculo das assinaturas.\newline
\par
Para iniciar este programa, basta executar o ficheiro \textbf{DirectoryTools.jar} presente na raíz do projeto submetido, clicando neste duas vezes. É, no entanto, necessário instalar no mínimo a versão 8 do Java (JRE). \newline

\par
Exite também um diretório de demonstração, \textbf{TestDirectory}, com vários ficheiros de tipos diferentes e um outro diretório, \textbf{TestFiles} com alguns ficheiros semelhantes aos do TestDirectory, para que se possam testar as funcionalidades da aplicação.

\section{Módulos}

\subsection{Counting Bloom Filter}
A implementação do Counting Bloom Filter foi feita de forma genérica de forma a garantir que funciona para qualquer tipos de dados (String, int, byte[], ...).
\par
Para além disso tanto o número de hash functions utilizadas, \textbf{k}, como o tamanho do filtro, \textbf{n}, são calculados de forma dinâmica em função do número de elementos que vão ser inseridos, \textbf{m}, e da probabilidade de ocorrência de falsos positivos desejada, \textbf{p}.

\[ k = \frac{n}{m} * ln(2) \]
\[ n = \frac{-m*ln(p)}{ln(2)^2}  \]\newline

\par
Desta forma, o número de funções de hash é ótimo e está garantida uma percentagem de falsos positivos muito próxima da indicada ao programa. Tanto na aplicação de uso conjunto como nos testes, a probabilidade usada é 0.1\%.

\par
Devido ao facto de se tratar de um Counting Bloom Filter, foram implementadas as funções remove() e count().

\subsection{Procura de elementos semelhantes (MinHash e Locality Sensitive Hashing)}

À semelhança do Bloom Filter, este módulo também é genérico, funcional para qualquer tipo de dados.
\par
Já que a vasta maioria de ficheiros é composta por milhares ou milhões de bytes, as shingles de um elemento são geradas utilizando um comprimento de 15 bytes. 
\par
Em vez de usar uma matriz de shingles, a assinatura de cada elemento é calculada (com o algoritmo de MinHash e 300 Hash Functions) enquanto este é adicionado.

O Locality Sensitive Hashing (LSH) divide a assinatura do elemento em 20 bandas de 15 rows cada.
Assim, elementos com semelhança 0.6 têm menos de 0.1\% de probabilidade de serem considerados candidatos, enquanto que elementos com semelhança 0.9 têm mais de 0.99\%.
\[ S = (\frac{1}{b})^{\frac{1}{r}} = 0.819\]
\subsection{Contador Estocástico}

Foi utilizada uma variante do contador estocástico que incrementa com uma probabilidade de \(\frac{1}{16}\), de forma a aumentar o número até ao qual se pode contar, mas sem que o erro na aproximação aumente de forma significativa.

\subsection{Funções de Hash}

Nos primeiros 2 módulos todas as hash functions utilizadas são geradas através de \textbf{Universal Hashing}, da seguinte forma:
\[ f(x) = (ax + b) \:mod\:  p \]
Em que \textbf{a} e \textbf{b} representam números gerados aleatóriamente e \textbf{p} um número primo 'grande' (exemplo: 1086216277). Cada função tem valores de \textbf{a} e \textbf{b} diferentes.
Assim garante-se que as funções não são correlacionadas.



%%%%%%%%%%%%%%%%%%%%%%%%%%%%%%%%
\chapter{Testes}
\label{chap.Testes}
\section{CountingBloomFilter}
Para correr o Teste do Counting Bloom Filter abre-se o terminal e executa-se o TestBloomFilter.jar:
\\
\textit{ java -jar TestBloomFilter.jar} (mínimo Java 8)

Foram Criados 3 Filtros
\subsubsection{Filtro 1}
No primeiro filtro adicionam-se 1000 Strings aleatórias com 40 caracteres.
Gerou-se 10000 novas strings aleatórias e verifica-se se pertencem ao bloom filter
A probabilidade de qualquer das novas strings é muito baixa(por ter 40 caracteres aleatórios), por isso, todas as strings que passem no teste são contadas como falsos positivos.

\subsubsection{Filtro 2}
No segundo filtro adicionaram-se 3 Strings aleatórias, também de 40 caracteres.
Verifica-se se 2 das strings adicionadas estão no filtro.
Verifica-se que uma strings diferente não está no filtro.
Remove-se uma das strings e verifica-se se está no filtro.

\subsubsection{Filtro 3}

No terceiro filtro adicionaram-se 100 Strings aleatórias em que uma delas aparece várias vezes no filtro.
Verifica-se quantas vezes essa string se encontra no filtro.
Remove-se algumas vezes essa string e voltamos a ver a contar essa string no filtro e verificamos que a diferença se verifica.

\section{SimilarFinder}
Para correr o Teste da MinHash abre-se o terminal e executa-se o TestSimilarFinder.jar:
\\
\textit{ java -jar TestSimilarFinder.jar} (mínimo Java 8)


Lê-se um ficheiro com várias as 5 primeiras estrofes dos Lusíadas e 5 strings para teste.
Verifica-se a existência de de strings semelhantes a "\textit{As armas e os Barões assinalados}".
As Strings candidatas pelo Locality Sensitive Hashing são diferentes em apenas um caracter.

O Min Hash verifica destas strings qual a mais próxima e calcula a diferença de Jaccard.

De seguida verifica a existência de strings duplicadas. No ficheiro encontra-se um string duplicada em que é verificada a sua distância de Jaccard (0).

\subsection{Locality Sensitive Hashing} 
Adiciona-se 10 mil strings aleatórias com 40 caracteres.
Adicionam-se mais 1000 strings aleatórias e para cada uma é contado o número de strings(das originalmente adicionadas) candidatas a serem semelhantes usando o Locality Sensitive Hashing. O ideal é não existir candidatos.

\section{StochasticCounter}
Para correr o Teste da Contagem Estocástica abre-se o terminal e executa-se o TestStochasticCounter.jar:
\\
\textit{ java -jar TestStochasticCounter.jar} (mínimo Java 8)

Realizamos 3 testes com o contador cada um para probabilidades diferentes(1/2, 1/16 e 1/32).

Testamos o contador para valores de base 10 (10, 100, 1000, até 100000000);
Em cada valor verificamos o valor que é registado pelo contador, aproximamos à ao valor real.
Vemos a diferença e calculamos o erro. Quanto mais alto é o valor real menor a probabilidade erro.

\section{DirectoryTools}
\par É um teste automático da aplicação de uso conjunto.
Para correr o Teste  abre-se o terminal e executa-se o TestDirectoryTools.jar:
\\
\textit{ java -jar TestDirectoryTools.jar} (mínimo Java 8)


Lê-se todos os ficheiros presentes no diretório "TestDirectory" e calcula-se o seu tamanho através da Contagem Estocástica.

Testamos se outros ficheiros do diretório TestFiles pertencem ou se são parecidos aos do diretório de teste e calcula-se a respetiva distância de Jaccard. Verifica-se o mesmo com ficheiros removidos.
Faz-se a contagem de quantas vezes os ficheiros são adicionados.
Deteta-se quais os ficheiros que estão duplicados.



\end{document}
