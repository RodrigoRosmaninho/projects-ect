\documentclass{report}
\usepackage[T1]{fontenc} % Fontes T1
\usepackage[utf8]{inputenc} % Input UTF8
\usepackage[backend=biber, style=ieee]{biblatex} % para usar bibliografia
\usepackage{csquotes}
\usepackage[portuguese]{babel} %Usar língua portuguesa
\usepackage{blindtext} % Gerar texto automaticamente
\usepackage[printonlyused]{acronym}
\usepackage{hyperref} % para autoref
\usepackage{graphicx}
\usepackage{ragged2e}
\usepackage{minted} % syntax highligthing
\bibliography{bibliografia}

\begin{document}

% Definições %
\def\titulo{CRAL}
\def\subtitulo {CiberRato Agent Language\vspace{3mm}\\Linguagem para definição e manipulação de robots}
\def\data{Julho 2019}
\def\autores{Eurico Dias, Pedro Valério, Daniel Correia, Rodrigo Rosmaninho, Rita Amante}
\def\autorescontactos{(72783) dias.eurico@ua.pt, (88734) pedrovalerio@ua.pt, (88753) dcorreia@ua.pt, (88802) r.rosmaninho@ua.pt, (89264) rita.amante@ua.pt}
\def\versao{VERSAO 1}
\def\departamento{Departamento de Eletrónica, Telecomunicações e Informática}
\def\empresa{Linguagens Formais e Autómatos}
\def\logotipo{ua.pdf}

% CAPA %
\begin{titlepage}
\begin{center}
\vspace*{50mm}
{\Huge \textbf{\titulo}}\\ 
\vspace{7mm}
{\Large\textbf{\subtitulo}}\\
\vspace{10mm}
{\Large \empresa}\\
\vspace{10mm}
{\LARGE \autores}\\ 
\vspace{20mm}

\begin{figure}[h]
\center
\includegraphics{\logotipo}
\end{figure}
\end{center}
\end{titlepage}

% Página de Título %
\title{
    {\Huge\textbf{\titulo}}\\
    \vspace{3mm}
    {\Large\text{\subtitulo}}\\
    \vspace{10mm}
    {\large \empresa \\ \departamento}
}
%\center
\author{
    \begin{center}
        \center
        \autores
    \end{center} 
    \center
    {\autorescontactos
    \date{\data}\\}
}
%\begin{center}
%\end{center}
\maketitle

\tableofcontents
% \listoftables     % descomentar se necessário
% \listoffigures    % descomentar se necessário


\clearpage
\pagenumbering{arabic}

% CAPITULO 1 - Objetivos %
\chapter{Objetivos}
\label{chap.objectives}

O principal objetivo deste projeto é desenvolver uma linguagem que facilite a programação de robots, com especial foco na programação de agentes virtuais no ambiente de simulação do \textsl{Ciber Rato}.

O \textsl{Ciber Rato}  é uma Modalidade de Simulação do Concurso Micro-Rato da Universidade de Aveiro e do seu IEEE Student Branch, que também é utilizada como atividade na Academia de Verão para estudantes do ensino secundário. Esta modalidade é suportada por um ambiente de simulação de robots e labirintos.

Aos concorrentes é proposto o desafio de construir uma simulação de um robot móvel e autónomo que resolva um labirinto desconhecido, no sentido de localizar e alcançar um farol nele colocado e imobilizar-se numa “zona de chegada”. Uma vez alcançado o primeiro objetivo o robot deverá regressar ao seu ponto de partida. Durante o percurso os robots encontram obstáculos fixos - paredes - e obstáculos móveis - outros robots - com os quais não podem colidir, sob pena de serem penalizados.

São disponibilizadas bibliotecas para interagir com a simulação e controlar os robots em \textbf{Python}, \textbf{Java}, e \textbf{C++}. No entanto, os participantes que frequentam o ensino secundário mostram alguma dificuldade a implementar os algoritmos devido ao facto de ser, para muitos, o primeiro contacto com programação em geral. 

Como tal, surge a necessidade de uma nova linguagem que simplifique o processo. Os objetivos principais são:
\begin{itemize}
    \item Simplificar a sintaxe em geral;
    \item Permitir a utilização de intervalos, por exemplo "[3, 5]" ou "3 < x < 7";
    \item Apresentar mensagens de erro muito informativas;
    \item Disponibilizar funcionalidades específicas à robótica;
    \item Permitir a configuração de vários aspetos do compilador de forma fácil para acomodar alterações ao concurso.
\end{itemize}

% CAPITULO 2 - Instruções de Utilização %
\chapter{Instruções de Utilização}
\label{chap.instructions}

Para executar o processo de compilação, é necessário correr os seguintes comandos a partir da raiz do repositório:
\begin{minted}{bash}
javac lib/*.java language/*.java config/*.java
java -ea language/cralMain $nome_do_ficheiro_cral $nome_do_ficheiro_config

# Exemplo com ficheiros reais
java -ea language/cralMain tests/a0.cral config/example.config
\end{minted}

O ficheiro compilado irá ser criado na raiz do repositório com o nome \textbf{Output.cpp}\\

No diretório \textbf{tests} existem alguns ficheiros de teste das funcionalidades base da linguagem e o ficheiro \textbf{a0.cral}, que consiste numa implementação em \textit{CRAL} do ficheiro \textbf{a0.cpp} que foi fornecido com as bibliotecas de interação com o simulador e, como tal, tem o mesmo comportamento e pode ser testado com o simulador.

No diretório \textbf{config} existem dois ficheiros exemplo de configuração que possibilitam a interação com a biblioteca \textbf{CiberAV}. Desses, o \textbf{example.config} é indicado para testes que interajam com o simulador, e o \textbf{example\_no\_init.config} para restantes testes de funcionalidade básica da linguagem.

% CAPITULO 3 - Linguagem %
\chapter{Linguagem}
\label{chap.language}
\textit{CRAL} é uma linguagem \textit{domain-specific} compilada cuja linguagem destino é \textbf{C++}, de forma a tirar partido das bibliotecas existentes para interagir com a simulação do \textit{Ciber Rato}.
A análise sintática e semântica está implementada em \textbf{ANTLR4}.\\

Exemplo de um programa muito simples implementado em \textit{CRAL}:
\begin{minted}{python}
    cond left_of_start = call startAngle() < -5
    cond right_of_start = call startAngle() > 5
    cond at_start = call startDistance() < 50

    behav gotoStart() until at_start:
        # going to target, adjusting orientation if necessary
        when:
            left_of_start: call motors(left=80, right=70)
            right_of_start: call motors(left=70, right=80)
            other: call motors(left=80, right=80)
        end
    end
    
    main():
        # connecting to server
        name = "Grimmy Bear"
        
        set goToStart
        
        # ending the program
        call print("Bye!")
    end
\end{minted}

\section{Instruções}
Em \textit{CRAL} existem vários tipos de instruções.

\subsection{Instruções de uma linha}

Algumas instruções, como, por exemplo, declarações e atribuições, ocupam apenas uma linha.

\begin{minted}{python}
bool verdadeiro         # Declaração
verdadeiro = true;      # Atribuição - ponto e virgula (';') é opcional
num numero = 1999       # Declaração e Atribuição simultaneamente
\end{minted}

No entanto, é possível condensar várias instruções numa única linha, desde que seja usado o separador '\textbf{;}'
\begin{minted}{python}
bool verdadeiro; verdadeiro = true; num numero = 1999
\end{minted}

É permitido a utilização de '\textbf{;}' para marcar o fim de uma instrução, embora não seja necessário se a instrução for a única na linha onde foi escrita.

\subsection{Instruções que podem gerar um bloco}
Certas instruções, como as de controlo de fluxo, podem gerar um bloco de instruções.

No corpo dos blocos pode estar qualquer instrução exceto a definição de uma função ou behaviour.
Podem também existir blocos adicionais dentro de um bloco.

Um bloco é iniciado pelo caráter '\textbf{:}' e terminado por '\textbf{end}'.

\begin{minted}{python}
call print("fora do bloco")
while true:
    call print("dentro do bloco")
    until x > 3:
        call print("bloco ao quadrado :D")
    end
end
\end{minted}


\subsection{Comentários}
O código pode ser comentado para efeitos de documentação. Os comentários não são executados e, como tal, é permitida a escrita livre.

Para iniciar um comentário de uma linha utiliza-se o caráter '\textbf{\#}', após o qual todo o texto até à próxima linha é ignorado pelo processo de compilação

Para escrever um comentário multi-linha delimita-se o texto pretendido pelos carateres '\textbf{\#*}' e '\textbf{*\#}', respetivamente.

\begin{minted}{python}
# Comentário de uma linha
call print("olá")
call print("hello") # Outro Comentário de uma linha
call print("bonjour")

#* 
linha 1
linha 2
etc
*#

call print("adeus") #* ou apenas uma *#
\end{minted}

\subsection{Calls}
\textbf{Calls} são funções disponíveis para utilizar e definidas no ficheiro de configuração, que permitem ao programador de \textit{CRAL} interagir com as funções da(s) biblioteca(s) de interação com a simulação/robots.

O compilador fornecido aos programadores possui listado no ficheiro de configuração as bibliotecas a importar e as funções que estão disponíveis para utilizar, tal como o seu tipo de retorno e argumentos, de modo a poder gerar erros semânticos se estas forem usadas de forma errada. Este ficheiro de configuração é alterável. É possível consultar mais informações sobre este tópico no \textit{\autoref{chap.config}}.\\

Do ponto de vista do programador as calls funcionam exatamente como funções normais, mas chamá-las requer a keyword \textbf{call}.

\begin{minted}{python}
    call print("hi")
    call motors(left=70, right=50)
    num angle = call beaconAngle()
\end{minted}

\section{Palavras Reservadas}
Listagem das palavras reservadas de \textit{CRAL}. Estas não podem ser usadas como nomes de variáveis/funções/behaviours.
\vspace{5mm}
\begin{center}
\begin{tabular}{| c | c | c | c |}
  \hline
  when & while & for & until\\
  \hline
  local & behav & set & end\\
  \hline
  string & num & cond & bool\\
  \hline
  break & return & other & call\\
  \hline
  true & false & &\\
  \hline
  \end{tabular}
\end{center}
\vspace{3mm}

Para além destas, não é permitido definir funções com nomes idênticos a \textbf{calls} definidas no ficheiro de configuração [\ref{chap.config}].

\section{Tipos}
Estão disponíveis os seguintes tipos:
\subsubsection{num (number)}\\
    \\Número real inteiro ou decimal. Utilizado para representar quantidades reais. Permite valores positivos e negativos.
\subsubsection{string}\\
    \\Utilizado para representar texto, equivalente a String em \textbf{C++}. O valor literal representa-se entre aspas '"'.
    
    Permite concatenação de expressões de qualquer tipo de forma simples. Utilizando o token '\textbf{\$}' como no exemplo seguinte:
    
    \begin{minted}{python}
    num fd = 10
    num ba = 90
    
    num sum(num a, num b): 
        return a + b
    end
    
    call print("distances: $fd$, $3+2$, $sum(5,10)$; angle: $ba$")
    # deverá ser impressa a string: distances: 10, 5, 15; angle: 90
    \end{minted}
    
    Se se pretender escrever o caracter literal '\textbf{\$}' na string é necessário escapá-lo utilizando '\textbf{\textbackslash\$}'.
    \begin{minted}{python}
    num x = 1
    string y = "Writing \$x\$ in this string results in: $x$"
    # Output: "Writing $x$ in this string results in: 1"
    \end{minted}
    
\subsubsection{cond}\\
    \\Condição cujo valor booleano (\textbf{true} ou \textbf{false}) é re-calculado cada vez que se pretende aceder a ele.
    Equivalente a uma função booleana mas com sintaxe mais simples.
    
    Este tipo de variável é \textit{final}, ou seja, uma vez configurada, não se pode modificar novamente, funcionando, assim, como uma constante. \\
    
    \textit{Contextualização:} 
Nos programas desenvolvidos pelos participantes da Academia de Verão em \textbf{C++} é frequente a utilização de estruturas \textbf{if-else if-else} de grandes dimensões em que algumas das condições são reutilizadas em outras partes do programa.

Por exemplo, verificar se a distância a um obstáculo está dentro de um certo intervalo. Se for necessário alterar o intervalo o aluno terá de o fazer em diversos locais no código e tende a confundir-se.

Dado este problema a equipa decidiu disponibilizar o tipo \textbf{cond}, que permite atribuir um label a uma condição booleana e utilizá-lo no resto do código. Assim, se for necessária uma alteração da condição, esta faz-se apenas na declaração da variável \textbf{cond}. Para além disso, a leitura do código torna-se mais fácil.

\subsubsection{bool}\\
    \\Valor booleano normal (valor permanece estático). As expressões são calculadas e o valor final pode ser \textbf{true} ou \textbf{false}.

\section{Variáveis}
Podem ser declaradas variáveis de qualquer tipo suportado pela linguagem, sendo que este tem de ser explicitamente indicado e não pode ser alterado durante o tempo de vida da variável.

Por omissão todas as variáveis declaradas pertencem ao contexto global, exceto se for utilizada a keyword \textbf{local}. Desta forma alivia-se a confusão sentida pelos participantes que têm dificuldade em perceber programação por contextos.

\begin{minted}{python}
# Exemplos de declarações
num answer
string nome
# Declarações com assign
cond at_start = startDistance() < 50
local bool verdadeiro = false   # Variável local
# Assign de variaveis já declaradas
answer = 42
verdadeiro = true
\end{minted}

\section{Expressões e Operadores}

Listagem dos operadores disponíveis para usar em expressões:

\begin{center}
\begin{tabular}{| c | c | c |}
  \hline
  \textbf{Operador} & \textbf{Função} & \textbf{Tipos suportados}\\
  \hline
  + & soma aritmética & num\\
  \hline
  - & subtração aritmética | sinal negativo & num\\
  \hline
  * & multiplicação aritmética & num\\
  \hline
  / & divisão aritmética & num\\
  \hline
  \% & resto da divisão inteira & num\\
  \hline
  \^{} & potência & num\\
  \hline
  ++ &  incremento por 1 & num\\
  \hline
  -{}- & decremento por 1 & num\\
  \hline
  += & operador de atribuição da adição & num\\
  \hline
  -= & operador de atribuição da subtração & num\\
  \hline
  *= & operador de atribuição da multiplicação & num\\
  \hline
  /= & operador de atribuição da divisão & num\\
  \hline
  \%= & operador de atribuição do resto & num\\
  \hline
  \^{}= & operador de atribuição da potência & num\\
  \hline
  is & igualdade & num, string, bool, cond\\
  \hline
  == & igualdade (símbolo alternativo) & num, string, bool, cond\\
  \hline
  != & desigualdade & num, string, bool, cond\\
  \hline
  > & maior & num\\
  \hline
  < & menor & num\\
  \hline
  >= & maior ou igual & num\\
  \hline
  <= & menor ou igual & num\\
  \hline
  and & conjunção lógica & bool, cond\\
  \hline
  or & disjunção lógica & bool, cond\\
  \hline
  not & negação lógica & bool, cond\\
  \hline
  in [x,y] & pertença a um intervalo & num\\
  \hline
  \end{tabular}
\end{center}

\vspace{3mm}

\begin{center}
\begin{tabular}{| c | c |}
  \hline
  \textbf{Notação de intervalo} & \textbf{Tradução}\\
  \hline
  x in [-1,2] & -1 <= x <= 2\\
  \hline
  x in ]-1,2] & -1 < x <= 2\\
  \hline
  x in [-1,2[ & -1 <= x < 2\\
  \hline
  x in ]-1,2[ & -1 < x < 2\\
  \hline
  \end{tabular}
\end{center}

\vspace{4mm}

É também permitida a utilização de expressões booleanas ternárias de verificação de pertença a um intervalo:
\begin{minted}{python}
-1 < x <= 2  # Equivale a: x > -1 and x <= 2
\end{minted}

\begin{minted}{python}
# Exemplos de expressões
num expr = 3 + 4
bool outra_expr = 30 > 4 and false
expr += 2   # expr = 9
while expr in [9, 15[ :
    expr++
end
# expr = 15
\end{minted}

\section{Controlo de Fluxo}
\subsection{Escolha - when blocks}
\textit{Contextualização:} 
Nos programas desenvolvidos pelos participantes da Academia de Verão em \textbf{C++} é frequente a utilização de estruturas \textbf{if-else if-else} de grandes dimensões e em que cada condicional executa um número reduzido de instruções. Pelo que seria conveniente simplificar a sintaxe.\\ Dado este problema a equipa decidiu tentar emular a sintaxe de um bloco \textbf{switch-case}.\\

Para iniciar um \textbf{when-block} utiliza-se a expressão \textbf{when} seguida de '\textbf{:}', e \textbf{end} para terminar. Dentro do bloco apenas é permitido escrever condições. A 'cabeça' da condição é uma expressão ou variável do tipo \textbf{bool} ou \textbf{cond}. O corpo é uma sequência de instruções que podem ser in-line (com uso de '\textbf{;}' caso seja mais do que uma) ou multi-line.

Qualquer condição tem prioridade sobre as condições escritas posteriormente e \textbf{other} é uma palavra reservada que equivale a \textbf{else}. Pelo que:

\begin{minted}{python}
when:
    x > 1: call print("if")
    bool_variable: call print("else if")
    other: call print("else")
end
\end{minted}

Equivale, em \textbf{C}, a:\\

\begin{minted}{C}
if(x > 1) {
    print("if"); // printf(......)
}
else if(bool_variable){
    print("else if");
}
else{
    print("else");
}
\end{minted}\\

\vspace{7mm}

\begin{minted}{python}
# Exemplo de when-block
when:
    # Equivalente, em C, a: if(cond_variable) { num = 42; }
    cond_variable: num = 42
    bool_variable: call print("else if")
    x in [3,8]: motors(30,50)
    x > 50: 
        call print("else if")
        while true:
            call print("infinite loop")
        end
    other: call print("else")
end
\end{minted}\\
\vspace{5mm}

\subsection{Ciclos}
Em \textit{CRAL} são disponibilizados 3 tipos de ciclos distintos: 
\begin{itemize}
    \item \textbf{while}\\
    \\ Este tipo de ciclo é equivalente a um \textbf{while loop} em \textbf{C++}. O ciclo acontece enquanto a condição for verdadeira. Aceita qualquer condição booleana (\textbf{bool} ou \textbf{cond}).
    \begin{minted}{python}
        # Exemplo de ciclos while
        while true:
            while 3 > 6:
                call print("unreachable code")
            end
        end
        
        num soma = 0
        while bool_or_cond_variable:
            soma = soma + 2
        end
    \end{minted}
    \item \textbf{for}\\
    \\ Este tipo de ciclo é equivalente a um \textbf{for loop} em \textbf{C++}. Pode ser utilizado de 2 formas diferentes.
    \begin{enumerate}
        \item \textbf{for} de 3 expressões (inicialização, condição, ação posterior)\\
        \\As três expressões são separadas por '\textbf{;}' e significam, respetivamente, ações que são executadas antes do início do ciclo, condição que é verificada a cada iteração, e ações que são efetuadas no fim de cada iteração.\\ 
        \begin{minted}{python}
        for num i = 0; i < 4; i = i + 1:
            call print(i) # É executado 3 vezes
        end
        \end{minted}\\
        \vspace{3mm}
        \item \textbf{for} num intervalo
        \begin{minted}{python}
        for num i in [0, 3]:
            call print(i) # É executado 3 vezes
        end
        \end{minted}
    \end{enumerate}
    
    Ambos os ciclos \textbf{for} apresentados têm o mesmo resultado/output.
    \item \textbf{until}\\
    \\ Este tipo de ciclo é equivalente a um \textbf{while loop} em \textbf{C++} com a sua condição negada. Isto é, o ciclo acontece até que a condição se torne verdadeira. Aceita qualquer condição booleana (\textbf{bool} ou \textbf{cond}).
    \begin{minted}{python}
        # Exemplo de ciclos until
        num soma = 0
        until soma > 10:
            soma = soma + 1
        end
        call print(soma) # soma = 11
    \end{minted}
\end{itemize}

\subsection{Funções}

Um programa em \textit{CRAL}, à semelhança de outras linguagens como o \textbf{C}, pode ser organizado em funções.
As funções possuem obrigatoriamente um nome e um bloco de código que é executado no caso de ser chamada. Podem adicionalmente ter uma lista de argumentos e um tipo de retorno. 
Uma função sem tipo de retorno é equivalente a uma função \textbf{void} em \textbf{C} e linguagens semelhantes.

Para tornar o código mas legível para os participantes da Academia de Verão, a lista de argumentos requer que o nome do argumento seja identificado. 

As funções declaradas podem ser chamadas com a sintaxe apresentada no exemplo que se segue.

\begin{minted}{python}
sayHi():
    call print("Hi")
end

num soma(num op1, num op2):
    return op1 + op2                # Retornar num
end

string getNome():
    return "\textit{CRAL}"            # Retornar String
end

main():
    #* Chamar Funções *#
    # Utilização sem valor de retorno ou argumentos
    sayHi()
    # Utilização do valor de retorno
    call print(getNome())
    # Utilização do valor de retorno e argumentos
    num resposta = soma(op1=20, op2=21) + 1
    # A ordem em que os argumentos aparecem não é importante
    num resposta2 = soma(op2=99, op1=1) + 1
end
\end{minted}

\subsection{Behaviours}
\label{behav}

\textit{Contextualização:} 
Nos programas desenvolvidos pelos participantes da Academia de Verão em \textbf{C++} é frequente a utilização de um elevado número de ciclos de grandes dimensões para descrever comportamentos do robot enquanto a condição do ciclo se verificar. Esta estrutura tem efeitos adversos na legibilidade do código.

Um \textbf{behaviour} é essencialmente uma função que executa o bloco em ciclo até uma condição se verificar. A cada ciclo de um behaviour é chamada a call apply que garante que o behaviour está sincronizado com simulador. 
Isto é, corresponde a um determinado comportamento do robot.

Desta forma, um programa pode ser abstraído como uma sequência de comportamentos.\\

Para iniciar um comportamento é utilizada a keyword \textbf{set}.

\begin{minted}{python}
at_beacon = call groundType() is target
left_of_beacon  = call beaconAngle(target) < -5
right_of_beacon = call beaconAngle(target) > 5

behav gotoBeaconAvoidingObstacles until at_beacon:

    # going to target, adjusting orientation if necessary
    # get sensor data and beacon angle
    num fd = call sensorData(FRONT)
    num ld = call sensorData(LEFT)
    num rd = call sensorData(RIGHT)
    num ba = call beaconAngle(target)
    call print("distances: $fd$, $ld$, $rd$; angle: $ba$")

    when:
        fd < 80:
            # nested when-block
            when:
                ld < 80: call motors(left=50,right=-50)
                other: call motors(left=-50, right=50)
            end
        ld < 80: call motors(left=80, right=40)
        rd < 80: call motors(left=40, right=80)
        left_of_beacon: call motors(left=80, right=40)
        right_of_beacon: call motors(left=40, right=80)
        other: call motors(left=80, right=80)
    end
end

main():
    num target = 1
    set gotoBeaconAvoidingObstacles
    # após o primeiro behav ter terminado, iniciar outro
    set anotherBehav
end
\end{minted}

% CAPITULO 4 - Configuração do Compilador %
\chapter{Configuração do Compilador}
\label{chap.config}
\section{Introdução}

É permitida a configuração de certos aspetos da compilação através da escrita de um ficheiro \textbf{.config}, que é interpretado apenas uma vez no início do processo de compilação.\\

\textit{Contextualização}: Visto que o objetivo da linguagem é a programação de robots reais e, principalmente, simulados, os programas feitos nesta necessitam de um conjunto de funções que lhe permitam interagir com o robot/simulador. Estas funções provêm essencialmente de bibliotecas externas em C++ que devem ser importadas no ficheiro output (também \textbf{C++}).

Portanto, é evidente a necessidade de haver um ficheiro de configuração onde os imports que o ficheiro output deve ter sejam especificados.
No entanto, essa configuração não é suficiente, já que o compilador de \textit{CRAL} desconheceria quais as funções disponíveis nas bibliotecas importadas e os seus argumentos e tipos de retorno. O que impossibilita a indicação de erros, caso existam.\\

O ficheiro inclui 3 componentes: \textbf{header}, \textbf{define}, e \textbf{calls}, por essa ordem. No entanto, apenas \textbf{calls} é obrigatória.
\begin{minted}{text}
# Exemplo
header :
#include <stdio.h>
#include "CiberAV.h"
end
define:
    FRONT : int
    LEFT : int
    RIGHT : int
    REAR : int
end

#* multi-line
   comment *#
calls:
    # inline comment
    call "init":
        init
        vars:
            "name" : string [1:] "Name can not be empty"
            "pos"  : int [:] ""
            "host" : string [:] ""
        end
        methods:
            [name, pos] : "init(<name>,<pos>)"
            [name, pos, host] : "init2(<name>,<pos>,<host>)"
        end
    end
    call "north":
        critical    # another comment
        return: int end
        methods:
            [] : "northAngle()"
        end
    end
end
\end{minted}

Tal como em \textit{CRAL}, os blocos iniciam-se com o caráter '\textbf{:}' e terminam com a keyword \textbf{end}.

\section{Header}
A componente \textbf{header} permite escrever código em \textbf{C++} que é injetado no topo do ficheiro output da compilação.
É especialmente útil para importar bibliotecas relevantes à programação do robot, interação com o simulador, entre outros.
Esta componente, ao contrário da componente \textbf{calls}, é opcional.

\begin{minted}{text}
# Exemplo
header :
#include <stdio.h>
#include "CiberAV.h"
// comentário em C++ aparece também no topo do ficheiro output
int func(int arg){
    return arg + 1;
}
end
\end{minted}

De notar que quaisquer funções definidas no \textbf{header} terão de constar em \textbf{calls} para que o compilador tome conhecimento delas e funcionem corretamente.

\section{Define}
A componente \textbf{define} permite listar as variáveis e constantes que se pretendem disponibilizar aos alunos existentes nas bibliotecas importadas ou no \textbf{header}. Desta forma, o compilador de \textit{CRAL} conhecerá quais as variáveis e constantes disponíveis nas bibliotecas importadas e os seus respetivos tipos.

\begin{minted}{text}
define:
    "FRONTSENSOR" : int
    "NAME" : string
    "val"  : double
    "is_finished" : bool
end
\end{minted}


\section{Calls}
Após inserir bibliotecas, é necessário definir todas as funções que o aluno poderá usar. Para tal, a componente \textbf{calls} apresenta uma lista das funções disponíveis, tal como o seu tipo de retorno e lista de argumentos, que permite posteriormente averiguar e indicar a ocorrência de erros, caso a função não tenha sido corretamente utilizada.

 Assim, a componente \textit{calls} contém uma lista de blocos \textbf{call} que representam funções, sendo que cada uma pode apresentar vários blocos como: \textit{init}, \textit{apply}, \textit{critical}, \textit{return}, \textit{vars} e \textit{methods}. Estes blocos têm que estar definidos por esta ordem, no entanto, só o bloco \textit{methods} é obrigatório.

\subsection{Init}
O bloco \textit{init} permite sinalizar a \textbf{call} que deve ser chamada em primeiro lugar na função \textit{main}, caso exista.

Este bloco não pode estar em simultâneo com o bloco \textit{apply}.\\

\textit{Contextualização}: Na biblioteca da Academia de Verão (CiberAV), que foi disponibilizada como referência para a realização deste projeto, existe uma função \textit{init} que é crucial ser chamada como primeira instrução da função \textit{main}. Por esse motivo, é importante o compilador tomar o conhecimento qual das funções definidas do bloco \textbf{calls} é esta função prioritária.

\begin{minted}{text}
# Exemplo
call "example":
    init
    # (.....)
end
\end{minted}

\subsection{Apply}
O bloco \textit{apply} tem como função identificar qual das calls é a função de atualização do simulador. (ver \textit{\autoref{behav}})

\begin{minted}{text}
# Exemplo
call "example":
    # (.....)
    apply
    # (.....)
end
\end{minted}

\subsection{Critical}
O bloco \textit{critical} significa qual a função que deve ser utilizada num contexto atualizado, como um \textbf{behaviour}. Permitindo assim apresentar um \textit{warning} ao programador se a função for utilizada em outros contextos.

\begin{minted}{text}
# Exemplo
call "example":
    # (.....)
    critical
    # (.....)
end
\end{minted}

\subsection{Return}
O bloco \textit{return} especifica o tipo de retorno da call, podendo tomar os valores: \textbf{int}, \textbf{double}, \textbf{bool}, e \textbf{string}.

\begin{minted}{text}
# Exemplo
call "example":
    # (.....)
    return: int end
    # (.....)
end
\end{minted}

\subsection{Vars}
O bloco \textit{vars} especifica as variáveis que podem ser passadas na chamada de uma call e as suas respetivas restrições, nomeadamente o seu tipo e, opcionalmente, valores permitidos e mensagem de erro apresentada caso as restrições não sejam respeitadas.
 
\begin{minted}{text}
# Exemplo
call "example":
    # (.....)
    vars: 
        "name" : string [1:] "Name can not be empty"
        "pos"  : int
        "left"  : int [-150:150] "power out of range"
        "sensor" : int [0:3] "sensor does not exist"
        "var" : int [0:3] "sensor does not exist"
    end
    # (.....)
end
\end{minted}

\subsection{Methods}
O bloco \textit{methods} contém as combinações das variáveis permitidas e a respetiva função a que correspondem no ficheiro importado no bloco \textbf{header}, em que os argumentos estão definidos no formato StringTemplate para facilitar o \textit{parsing} do ficheiro de configuração.

\begin{minted}{text}
# Exemplo
call "example":
    # (.....)
    methods:
        [name, pos] : "init(<name>,<pos>)"
        [name, pos, host] : "init2(<name>,<pos>,<host>)"
    end
end
\end{minted}

% Contribuições dos autores %
\chapter{Contribuições dos autores}
\label{chap.authors}
\begin{itemize}
    \item \textbf{Eurico Dias} - 20\%
        \begin{itemize}
        \item Compilador da linguagem \textit{CRAL}
        \item Ajuda no Ficheiro String Template Group para \textbf{C++}
        \item Classes da package lib
        \item Planeamento e conceptualização
        \end{itemize}
    \item \textbf{Pedro Valério} - 20\%
        \begin{itemize}
        \item Gramática da linguagem \textit{CRAL}
        \item Análise Semântica da linguagem \textit{CRAL}
        \item Classes da package lib
        \item Planeamento e conceptualização
        \end{itemize}
    \item \textbf{Daniel Correia} - 20\%
        \begin{itemize}
        \item Gramática do Ficheiro de Configuração 
        \item Interpretador do Ficheiro de Configuração
        \item Classes da package lib
        \item Planeamento e conceptualização
        \end{itemize}
    \item \textbf{Rodrigo Rosmaninho} - 20\%
        \begin{itemize}
        \item Ficheiro String Template Group para \textbf{C++}
        \item Relatório / Documentação
        \item Ajuda no Compilador da linguagem \textit{CRAL}
        \item Planeamento e conceptualização
        \end{itemize}
    \item \textbf{Rita Amante} - 20\%
        \begin{itemize}
        \item Error Handling
        \item Relatório / Documentação 
        \item Planeamento e conceptualização
        \end{itemize}
\end{itemize}

\end{document}